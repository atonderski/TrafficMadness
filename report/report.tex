\documentclass[12pt,a4paper]{article}

\usepackage[utf8]{inputenc}
\usepackage[T1]{fontenc}
\usepackage[swedish]{babel}
\usepackage{amsmath}
\usepackage{ae}
\usepackage{units}
\usepackage{icomma}
\usepackage{color}
\usepackage{graphicx}
\usepackage{bbm}
\newcommand{\N}{\ensuremath{\mathbbm{N}}}
\newcommand{\Z}{\ensuremath{\mathbbm{Z}}}
\newcommand{\Q}{\ensuremath{\mathbbm{Q}}}
\newcommand{\R}{\ensuremath{\mathbbm{R}}}
\newcommand{\C}{\ensuremath{\mathbbm{C}}}
\newcommand{\rd}{\ensuremath{\mathrm{d}}}
\newcommand{\id}{\ensuremath{\,\rd}}


\begin{document}

\title{}
\author{}
\date{}
\maketitle

\begin{abstract}

\end{abstract}

\newpage
\tableofcontents
\newpage

\section{}

\subsection{}

% Figurer inkluderade som eps-filer
%% \begin{figure}\centering
%% \includegraphics{filnamn.eps}
%% \caption{\label{figuren} Perioden $T$ som funktion av pendellängden.}
%% \end{figure}

% Figurer inkluderade med xfigs postscript+latex
%% \begin{figure}\centering
%% \input{filnamn.pstex_t}
%% \caption{\label{finafiguren} Perioden $T$ som funktion av
%%   pendellängden.}
%% \end{figure}

\end{document}
